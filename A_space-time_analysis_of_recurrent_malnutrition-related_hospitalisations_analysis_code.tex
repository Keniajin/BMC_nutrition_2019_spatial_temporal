\documentclass[]{article}
\usepackage{lmodern}
\usepackage{amssymb,amsmath}
\usepackage{ifxetex,ifluatex}
\usepackage{fixltx2e} % provides \textsubscript
\ifnum 0\ifxetex 1\fi\ifluatex 1\fi=0 % if pdftex
  \usepackage[T1]{fontenc}
  \usepackage[utf8]{inputenc}
\else % if luatex or xelatex
  \ifxetex
    \usepackage{mathspec}
  \else
    \usepackage{fontspec}
  \fi
  \defaultfontfeatures{Ligatures=TeX,Scale=MatchLowercase}
\fi
% use upquote if available, for straight quotes in verbatim environments
\IfFileExists{upquote.sty}{\usepackage{upquote}}{}
% use microtype if available
\IfFileExists{microtype.sty}{%
\usepackage{microtype}
\UseMicrotypeSet[protrusion]{basicmath} % disable protrusion for tt fonts
}{}
\usepackage[margin=1in]{geometry}
\usepackage{hyperref}
\hypersetup{unicode=true,
            pdftitle={Analysis Code},
            pdfborder={0 0 0},
            breaklinks=true}
\urlstyle{same}  % don't use monospace font for urls
\usepackage{color}
\usepackage{fancyvrb}
\newcommand{\VerbBar}{|}
\newcommand{\VERB}{\Verb[commandchars=\\\{\}]}
\DefineVerbatimEnvironment{Highlighting}{Verbatim}{commandchars=\\\{\}}
% Add ',fontsize=\small' for more characters per line
\usepackage{framed}
\definecolor{shadecolor}{RGB}{248,248,248}
\newenvironment{Shaded}{\begin{snugshade}}{\end{snugshade}}
\newcommand{\KeywordTok}[1]{\textcolor[rgb]{0.13,0.29,0.53}{\textbf{#1}}}
\newcommand{\DataTypeTok}[1]{\textcolor[rgb]{0.13,0.29,0.53}{#1}}
\newcommand{\DecValTok}[1]{\textcolor[rgb]{0.00,0.00,0.81}{#1}}
\newcommand{\BaseNTok}[1]{\textcolor[rgb]{0.00,0.00,0.81}{#1}}
\newcommand{\FloatTok}[1]{\textcolor[rgb]{0.00,0.00,0.81}{#1}}
\newcommand{\ConstantTok}[1]{\textcolor[rgb]{0.00,0.00,0.00}{#1}}
\newcommand{\CharTok}[1]{\textcolor[rgb]{0.31,0.60,0.02}{#1}}
\newcommand{\SpecialCharTok}[1]{\textcolor[rgb]{0.00,0.00,0.00}{#1}}
\newcommand{\StringTok}[1]{\textcolor[rgb]{0.31,0.60,0.02}{#1}}
\newcommand{\VerbatimStringTok}[1]{\textcolor[rgb]{0.31,0.60,0.02}{#1}}
\newcommand{\SpecialStringTok}[1]{\textcolor[rgb]{0.31,0.60,0.02}{#1}}
\newcommand{\ImportTok}[1]{#1}
\newcommand{\CommentTok}[1]{\textcolor[rgb]{0.56,0.35,0.01}{\textit{#1}}}
\newcommand{\DocumentationTok}[1]{\textcolor[rgb]{0.56,0.35,0.01}{\textbf{\textit{#1}}}}
\newcommand{\AnnotationTok}[1]{\textcolor[rgb]{0.56,0.35,0.01}{\textbf{\textit{#1}}}}
\newcommand{\CommentVarTok}[1]{\textcolor[rgb]{0.56,0.35,0.01}{\textbf{\textit{#1}}}}
\newcommand{\OtherTok}[1]{\textcolor[rgb]{0.56,0.35,0.01}{#1}}
\newcommand{\FunctionTok}[1]{\textcolor[rgb]{0.00,0.00,0.00}{#1}}
\newcommand{\VariableTok}[1]{\textcolor[rgb]{0.00,0.00,0.00}{#1}}
\newcommand{\ControlFlowTok}[1]{\textcolor[rgb]{0.13,0.29,0.53}{\textbf{#1}}}
\newcommand{\OperatorTok}[1]{\textcolor[rgb]{0.81,0.36,0.00}{\textbf{#1}}}
\newcommand{\BuiltInTok}[1]{#1}
\newcommand{\ExtensionTok}[1]{#1}
\newcommand{\PreprocessorTok}[1]{\textcolor[rgb]{0.56,0.35,0.01}{\textit{#1}}}
\newcommand{\AttributeTok}[1]{\textcolor[rgb]{0.77,0.63,0.00}{#1}}
\newcommand{\RegionMarkerTok}[1]{#1}
\newcommand{\InformationTok}[1]{\textcolor[rgb]{0.56,0.35,0.01}{\textbf{\textit{#1}}}}
\newcommand{\WarningTok}[1]{\textcolor[rgb]{0.56,0.35,0.01}{\textbf{\textit{#1}}}}
\newcommand{\AlertTok}[1]{\textcolor[rgb]{0.94,0.16,0.16}{#1}}
\newcommand{\ErrorTok}[1]{\textcolor[rgb]{0.64,0.00,0.00}{\textbf{#1}}}
\newcommand{\NormalTok}[1]{#1}
\usepackage{graphicx,grffile}
\makeatletter
\def\maxwidth{\ifdim\Gin@nat@width>\linewidth\linewidth\else\Gin@nat@width\fi}
\def\maxheight{\ifdim\Gin@nat@height>\textheight\textheight\else\Gin@nat@height\fi}
\makeatother
% Scale images if necessary, so that they will not overflow the page
% margins by default, and it is still possible to overwrite the defaults
% using explicit options in \includegraphics[width, height, ...]{}
\setkeys{Gin}{width=\maxwidth,height=\maxheight,keepaspectratio}
\IfFileExists{parskip.sty}{%
\usepackage{parskip}
}{% else
\setlength{\parindent}{0pt}
\setlength{\parskip}{6pt plus 2pt minus 1pt}
}
\setlength{\emergencystretch}{3em}  % prevent overfull lines
\providecommand{\tightlist}{%
  \setlength{\itemsep}{0pt}\setlength{\parskip}{0pt}}
\setcounter{secnumdepth}{0}
% Redefines (sub)paragraphs to behave more like sections
\ifx\paragraph\undefined\else
\let\oldparagraph\paragraph
\renewcommand{\paragraph}[1]{\oldparagraph{#1}\mbox{}}
\fi
\ifx\subparagraph\undefined\else
\let\oldsubparagraph\subparagraph
\renewcommand{\subparagraph}[1]{\oldsubparagraph{#1}\mbox{}}
\fi

%%% Use protect on footnotes to avoid problems with footnotes in titles
\let\rmarkdownfootnote\footnote%
\def\footnote{\protect\rmarkdownfootnote}

%%% Change title format to be more compact
\usepackage{titling}

% Create subtitle command for use in maketitle
\providecommand{\subtitle}[1]{
  \posttitle{
    \begin{center}\large#1\end{center}
    }
}

\setlength{\droptitle}{-2em}

  \title{Analysis Code}
    \pretitle{\vspace{\droptitle}\centering\huge}
  \posttitle{\par}
    \author{}
    \preauthor{}\postauthor{}
    \date{}
    \predate{}\postdate{}
  

\begin{document}
\maketitle

\subsection{Code 1 - Extract weather
data}\label{code-1---extract-weather-data}

Script for extracting MODIS vegetation indices for analysis

\begin{Shaded}
\begin{Highlighting}[]
\KeywordTok{set.seed}\NormalTok{(}\DecValTok{1221223}\NormalTok{)}
\KeywordTok{library}\NormalTok{(MODIS)}
\KeywordTok{library}\NormalTok{(dplyr)}
\KeywordTok{library}\NormalTok{(foreign)}
\KeywordTok{library}\NormalTok{(tidyr)}

\NormalTok{##read in the data with locations from KHDSS}
\NormalTok{##  1km by 1km latitude and longitude for the 1 by 1 km locations}
\NormalTok{## this file contains the dat[1] latitude and dat[2] longitude for the 1km by 1km locations }
\NormalTok{dat <-}\StringTok{ }\NormalTok{foreign}\OperatorTok{::}\KeywordTok{read.dta}\NormalTok{(}\StringTok{"data/KHDSS_1km_1km.DTA"}\NormalTok{)}

\CommentTok{#'         \textbackslash{}\textbackslash{}\textbackslash{}_Extract the rainfall data process_\textbackslash{}\textbackslash{}\textbackslash{}         #}
\CommentTok{#'#'\textbackslash{}\textbackslash{}_________Starting the extract of data_______________________\textbackslash{}\textbackslash{}}


\NormalTok{##identify the folder with the monthly  tiff data for EVI}
\NormalTok{## the files were downloaded earlier and saved }
\NormalTok{## https://modis.gsfc.nasa.gov/data/dataprod/mod13.php }
\NormalTok{##https://www.rdocumentation.org/packages/MODIS/versions/1.1.5/topics/runGdal}

\NormalTok{## alink that can help https://conservationecology.wordpress.com/2014/08/11/bulk-downloading-and-analysing-modis-data-in-r/}

\CommentTok{#'install https://www.gdal.org/}
\CommentTok{#'library(MODIS)}
\CommentTok{#'runGdal(product="MOD13Q1",begin=as.Date("01/02/2002",format = "%d/%m/%Y") ,}
\CommentTok{#'      end = as.Date("31/12/2015",format = "%d/%m/%Y"),extent="Kenya")}
\CommentTok{#'      }
\CommentTok{#'      'getTile("Kenya")  for tile specific}
\CommentTok{#'        ,tileH = 21:22,tileV = 8:9 }
\CommentTok{#'}

\CommentTok{#'(this process takes some time and needs 30 GB of space free to generate data for the whole admission period}
\NormalTok{vi <-}\StringTok{ }\KeywordTok{preStack}\NormalTok{(}\DataTypeTok{path =} \StringTok{"modis/monthly_data/"}\NormalTok{, }\DataTypeTok{pattern =} \StringTok{"*.tif$"}\NormalTok{)}

\NormalTok{### stack the data to data frame}
\NormalTok{s <-}\StringTok{ }\KeywordTok{stack}\NormalTok{(vi)}
\NormalTok{s <-}\StringTok{  }\NormalTok{s }\OperatorTok{*}\StringTok{ }\FloatTok{0.0001} \CommentTok{# Rescale the downloaded Files with the scaling factor (from modis) EVI}

\CommentTok{#'#'\textbackslash{}\textbackslash{}_________extracting for the 1km by 1km data_________\textbackslash{}\textbackslash{}}
\CommentTok{# And extract the mean value for our point from before.}
\CommentTok{# First Transform our coordinates from lat-long to to the MODIS sinus Projection}
\NormalTok{## method='bilinear' used for extraction }
\NormalTok{## If 'simple' values for the cell a point falls in are returned.}
\NormalTok{## If 'bilinear' the returned values are interpolated from the values of the four nearest raster cells.}

\NormalTok{sp <-}\StringTok{  }\KeywordTok{SpatialPoints}\NormalTok{(}\DataTypeTok{coords =} \KeywordTok{cbind}\NormalTok{(dat[}\DecValTok{2}\NormalTok{], dat[}\DecValTok{1}\NormalTok{]),}
    \DataTypeTok{proj4string =} \KeywordTok{CRS}\NormalTok{(}\StringTok{"+proj=longlat +datum=WGS84 +ellps=WGS84"}\NormalTok{) )}
\NormalTok{sp <-}\StringTok{ }\KeywordTok{spTransform}\NormalTok{(sp, }\KeywordTok{CRS}\NormalTok{(}\KeywordTok{proj4string}\NormalTok{(s)))}
\NormalTok{dataExtract <-}\StringTok{ }\NormalTok{raster}\OperatorTok{::}\KeywordTok{extract}\NormalTok{(s, sp , }\DataTypeTok{df=}\NormalTok{T, }\DataTypeTok{method=}\StringTok{"bilinear"}\NormalTok{) }\CommentTok{# Extract the EVI}
\KeywordTok{write.csv}\NormalTok{(dataExtract , }\StringTok{"data/modis_data_2001_2015.csv"}\NormalTok{)}

\CommentTok{#'@_______________________________________________________________}
\CommentTok{#'         \textbackslash{}\textbackslash{}\textbackslash{}_End Extract data process_\textbackslash{}\textbackslash{}\textbackslash{}         #}
\CommentTok{#'         }


\NormalTok{####}
\CommentTok{#'#'\textbackslash{}\textbackslash{}_________Extracting the rainfall data_________\textbackslash{}\textbackslash{}}
\NormalTok{###Extracting the rainfall data}
\NormalTok{##identify the folder with the monthly  tiff data for EVI}
\NormalTok{vi2 <-}\StringTok{ }\KeywordTok{preStack}\NormalTok{(}\DataTypeTok{path =} \StringTok{"modis/rainfall/"}\NormalTok{, }\DataTypeTok{pattern =} \StringTok{"resampledchirps-v2.0.20*"}\NormalTok{)}

\NormalTok{### stack the data to data frame}
\NormalTok{si2 <-}\StringTok{ }\KeywordTok{stack}\NormalTok{(vi2)}

\CommentTok{#'#'\textbackslash{}\textbackslash{}_________extracting for the admissions data_________\textbackslash{}\textbackslash{}}
\NormalTok{sp2_b <-}\StringTok{  }\KeywordTok{SpatialPoints}\NormalTok{(}\DataTypeTok{coords =} \KeywordTok{cbind}\NormalTok{(dat[}\DecValTok{2}\NormalTok{], dat[}\DecValTok{1}\NormalTok{]),}
                        \DataTypeTok{proj4string =} \KeywordTok{CRS}\NormalTok{(}\StringTok{"+proj=longlat +datum=WGS84 +ellps=WGS84"}\NormalTok{))}
\NormalTok{sp2_b <-}\StringTok{ }\KeywordTok{spTransform}\NormalTok{(sp2_b, }\KeywordTok{CRS}\NormalTok{(}\KeywordTok{proj4string}\NormalTok{(si2)))}
\NormalTok{dataExtractAdm2 <-}\StringTok{ }\NormalTok{raster}\OperatorTok{::}\KeywordTok{extract}\NormalTok{(si2, sp2_b , }\DataTypeTok{df=}\NormalTok{T, }\DataTypeTok{method=}\StringTok{"bilinear"}\NormalTok{) }\CommentTok{# Extract the rainfall}
\KeywordTok{write.csv}\NormalTok{(dataExtractAdm2 , }\StringTok{"dataExtractADM_rainfall.csv"}\NormalTok{)}
\CommentTok{#'@_______________________________________________________________}
\end{Highlighting}
\end{Shaded}

\subsection{Code 2: INLA Spatial Temproral
Models}\label{code-2-inla-spatial-temproral-models}

This code is for running INLA Spatial models

\begin{Shaded}
\begin{Highlighting}[]
\KeywordTok{set.seed}\NormalTok{(}\DecValTok{1221223}\NormalTok{)}
  \KeywordTok{rm}\NormalTok{(}\DataTypeTok{list =} \KeywordTok{ls}\NormalTok{())}

  \KeywordTok{library}\NormalTok{(R2WinBUGS)}
\NormalTok{  ##Mapping}
  \KeywordTok{library}\NormalTok{(rgeos)}
  \KeywordTok{library}\NormalTok{(maptools)}
  \KeywordTok{library}\NormalTok{(}\StringTok{"ggplot2"}\NormalTok{)}
  \KeywordTok{library}\NormalTok{(broom) ## for converting a map to a data frame}
  \CommentTok{#library(glm2)}
  \CommentTok{#library(ResourceSelection) ## for hosmer and lemeshow testing}
  \KeywordTok{library}\NormalTok{(dplyr)}
  \KeywordTok{library}\NormalTok{(INLA)}
  \KeywordTok{library}\NormalTok{(spdep)}
\NormalTok{  ## coloring the spplot}
  \KeywordTok{library}\NormalTok{(colorspace)}

\NormalTok{  ###reading and exporting the shape file}
\NormalTok{  ## shape file available upon request}
\NormalTok{  kilifi_sub <-}\StringTok{ }\NormalTok{maptools}\OperatorTok{::}\KeywordTok{readShapePoly}\NormalTok{ ( }\StringTok{"data/kilif_sub_loc_Shape/DSS_subloc_Arc.shp"}\NormalTok{,}
                                          \DataTypeTok{IDvar=}\StringTok{"Adj_ID"}\NormalTok{, }\DataTypeTok{proj4string=}\KeywordTok{CRS}\NormalTok{(}\StringTok{"+proj=longlat +ellps=clrk66"}\NormalTok{))}
  
\NormalTok{  temp <-}\StringTok{ }\NormalTok{spdep}\OperatorTok{::}\KeywordTok{poly2nb}\NormalTok{(kilifi_sub)}
  \KeywordTok{nb2INLA}\NormalTok{(}\StringTok{"data/kilif_sub_loc_Shape/DSS_subloc_Arc.graph"}\NormalTok{, temp)}
\NormalTok{  klf.adj <-}\StringTok{ }\KeywordTok{paste}\NormalTok{(}\KeywordTok{getwd}\NormalTok{(),}\StringTok{"/data/kilif_sub_loc_Shape/DSS_subloc_Arc.graph"}\NormalTok{,}\DataTypeTok{sep=}\StringTok{""}\NormalTok{)}
  
\NormalTok{  ### load the admissions data}
\NormalTok{  ## data available upon request}
\NormalTok{  admData <-}\StringTok{ }\KeywordTok{read.csv}\NormalTok{(}\StringTok{"data/morbidity.csv"}\NormalTok{)}
\NormalTok{  admData}\OperatorTok{$}\NormalTok{rain_mm <-}\StringTok{ }\NormalTok{admData}\OperatorTok{$}\NormalTok{rain_mm}\OperatorTok{/}\DecValTok{50}
\NormalTok{  admData}\OperatorTok{$}\NormalTok{severe_disease <-}\StringTok{ }\KeywordTok{factor}\NormalTok{(admData}\OperatorTok{$}\NormalTok{severe_disease , }\DataTypeTok{levels=}\KeywordTok{c}\NormalTok{(}\DecValTok{0}\NormalTok{,}\DecValTok{1}\NormalTok{,}\DecValTok{2}\NormalTok{,}\DecValTok{3}\NormalTok{))}
\NormalTok{  admData <-}\StringTok{ }\NormalTok{admData }\OperatorTok\StringTok{ }\KeywordTok{mutate}\NormalTok{(}\DataTypeTok{gender2=} \KeywordTok{ifelse}\NormalTok{(gender}\OperatorTok{==}\DecValTok{1}\NormalTok{ ,}\DecValTok{0}\NormalTok{,}\DecValTok{1}\NormalTok{ ))}

\NormalTok{  admData}\OperatorTok{$}\NormalTok{gender2 <-}\StringTok{ }\KeywordTok{factor}\NormalTok{(admData}\OperatorTok{$}\NormalTok{gender2 , }\DataTypeTok{levels=}\KeywordTok{c}\NormalTok{(}\DecValTok{0}\NormalTok{,}\DecValTok{1}\NormalTok{) )}
\NormalTok{  admData}\OperatorTok{$}\NormalTok{gender <-}\StringTok{ }\NormalTok{admData}\OperatorTok{$}\NormalTok{gender2}
\NormalTok{  admData2 <-}\StringTok{ }\NormalTok{admData }\OperatorTok\StringTok{  }\NormalTok{dplyr}\OperatorTok{::}\KeywordTok{select}\NormalTok{(Adj_ID , sublocation,mnth, nagem, gender , }
\NormalTok{                                         severe_disease ,}
\NormalTok{                                  cumulitive_count,cumulitive_time , EVI_VALUE ,count_adm ,rain_mm,}
\NormalTok{                                  total_admission ,admdays ,nweight ,yr)}
  
\NormalTok{  ###generate othe variables to be used within INLa}
\NormalTok{  admData2}\OperatorTok{$}\NormalTok{Adj_ID2 <-}\StringTok{ }\NormalTok{admData2}\OperatorTok{$}\NormalTok{Adj_ID}
\NormalTok{  admData2}\OperatorTok{$}\NormalTok{Adj_ID3 <-}\StringTok{ }\NormalTok{admData2}\OperatorTok{$}\NormalTok{Adj_ID}
\NormalTok{  admData2}\OperatorTok{$}\NormalTok{count_adm2 <-}\StringTok{ }\NormalTok{admData2}\OperatorTok{$}\NormalTok{count_adm}
\NormalTok{  admData2}\OperatorTok{$}\NormalTok{count_adm3 <-}\StringTok{ }\NormalTok{admData2}\OperatorTok{$}\NormalTok{count_adm}
\NormalTok{  admData2}\OperatorTok{$}\NormalTok{count_adm4 <-}\StringTok{ }\NormalTok{admData2}\OperatorTok{$}\NormalTok{count_adm}
\NormalTok{  admData2}\OperatorTok{$}\NormalTok{count_adm5 <-}\StringTok{ }\NormalTok{admData2}\OperatorTok{$}\NormalTok{count_adm}
\NormalTok{  admData2}\OperatorTok{$}\NormalTok{count_adm6 <-}\StringTok{ }\NormalTok{admData2}\OperatorTok{$}\NormalTok{count_adm}
\NormalTok{  admData2}\OperatorTok{$}\NormalTok{count_adm7 <-}\StringTok{ }\NormalTok{admData2}\OperatorTok{$}\NormalTok{count_adm}
\NormalTok{  admData2}\OperatorTok{$}\NormalTok{EVI_VALUE2 <-}\StringTok{ }\NormalTok{admData2}\OperatorTok{$}\NormalTok{EVI_VALUE}
\NormalTok{  admData2}\OperatorTok{$}\NormalTok{rain_mm2 <-}\StringTok{ }\NormalTok{admData2}\OperatorTok{$}\NormalTok{rain_mm}
\NormalTok{  admData2}\OperatorTok{$}\NormalTok{nagem2 <-}\StringTok{ }\NormalTok{admData2}\OperatorTok{$}\NormalTok{nagem}
\NormalTok{  admData2}\OperatorTok{$}\NormalTok{severe_disease2 <-}\StringTok{ }\KeywordTok{as.factor}\NormalTok{(admData2}\OperatorTok{$}\NormalTok{severe_disease)}
\NormalTok{  admData2}\OperatorTok{$}\NormalTok{mnth2 <-}\StringTok{ }\NormalTok{admData2}\OperatorTok{$}\NormalTok{mnth}
\NormalTok{  admData2}\OperatorTok{$}\NormalTok{nweight2 <-}\StringTok{   }\NormalTok{admData2}\OperatorTok{$}\NormalTok{nweight}
\NormalTok{  admData2}\OperatorTok{$}\NormalTok{admdays2 <-}\StringTok{   }\NormalTok{admData2}\OperatorTok{$}\NormalTok{admdays}
  
\NormalTok{  ###defining the priors}
\NormalTok{  prior.iid =}\StringTok{ }\KeywordTok{c}\NormalTok{(}\DecValTok{1}\NormalTok{,}\FloatTok{0.01}\NormalTok{)}
\NormalTok{  prior.besag =}\StringTok{ }\KeywordTok{c}\NormalTok{(}\DecValTok{1}\NormalTok{,}\FloatTok{0.001}\NormalTok{)}
\NormalTok{  initial.iid =}\StringTok{ }\DecValTok{4}
\NormalTok{  initial.besag =}\StringTok{ }\DecValTok{3}

  \CommentTok{#'         \textbackslash{}\textbackslash{}\textbackslash{}_Model_1_\textbackslash{}\textbackslash{}\textbackslash{}         #  }
\NormalTok{###############MOdel 1 #######################################################    }
\NormalTok{  ## spatial unstructured}
\NormalTok{  formulaUH0 <-}\StringTok{ }\NormalTok{cumulitive_count }\OperatorTok{~}\StringTok{ }\NormalTok{EVI_VALUE }\OperatorTok{+}\StringTok{ }\NormalTok{rain_mm  }\OperatorTok{+}\StringTok{ }\NormalTok{gender }\OperatorTok{+}\StringTok{ }\NormalTok{severe_disease }\OperatorTok{+}
\StringTok{    }\NormalTok{total_admission  }\OperatorTok{+}\StringTok{ }\NormalTok{admdays }\OperatorTok{+}\StringTok{ }\NormalTok{nweight }\OperatorTok{+}
\StringTok{    }\KeywordTok{f}\NormalTok{(Adj_ID, }\DataTypeTok{model =} \StringTok{"iid"}\NormalTok{,}\DataTypeTok{prior=}\StringTok{"normal"}\NormalTok{,}\DataTypeTok{param=}\KeywordTok{c}\NormalTok{(}\DecValTok{0}\NormalTok{, }\FloatTok{0.001}\NormalTok{) , }\DataTypeTok{initial =} \DecValTok{1}\NormalTok{)}
  
\NormalTok{  resultUH0 <-}\StringTok{ }\KeywordTok{inla}\NormalTok{(formulaUH0,}\DataTypeTok{family=}\StringTok{"nbinomial"}\NormalTok{,}
                   \DataTypeTok{data=}\NormalTok{admData2, }\DataTypeTok{control.compute=}\KeywordTok{list}\NormalTok{(}\DataTypeTok{dic=}\OtherTok{TRUE}\NormalTok{,}\DataTypeTok{cpo=}\OtherTok{TRUE}\NormalTok{),}\DataTypeTok{E=}\KeywordTok{log}\NormalTok{(nagem) ,}
                   \DataTypeTok{control.predictor =} \KeywordTok{list}\NormalTok{(}\DataTypeTok{compute =} \OtherTok{TRUE}\NormalTok{))}
  
\NormalTok{##summary in 3 decimal places}
  \KeywordTok{summary}\NormalTok{(resultUH0)}
  \KeywordTok{exp}\NormalTok{(resultUH0}\OperatorTok{$}\NormalTok{summary.fixed)}
  \KeywordTok{write.csv}\NormalTok{(}\KeywordTok{data.frame}\NormalTok{(resultUH0}\OperatorTok{$}\NormalTok{summary.fixed), }\StringTok{"results1_14504_36.csv"}\NormalTok{)}
  
\NormalTok{pdresultUH0 <-}\StringTok{ }\NormalTok{resultUH0}\OperatorTok{$}\NormalTok{dic}\OperatorTok{$}\NormalTok{p.eff}
  

 \CommentTok{#'         \textbackslash{}\textbackslash{}\textbackslash{}_Model_1_\textbackslash{}\textbackslash{}\textbackslash{}         #  }
 \CommentTok{#'         #}
  
  \CommentTok{#'         \textbackslash{}\textbackslash{}\textbackslash{}_Model 1B\textbackslash{}\textbackslash{}\textbackslash{}         #  }
\NormalTok{  ###############MOdel 1B#######################################################  }
\NormalTok{  ### spatial model structured and unstrustured  without }
\NormalTok{  ### to comapare with Winbugs}
\NormalTok{  formulaUHB <-}\StringTok{ }\NormalTok{cumulitive_count }\OperatorTok{~}\StringTok{ }\NormalTok{EVI_VALUE }\OperatorTok{+}\StringTok{ }\NormalTok{rain_mm   }\OperatorTok{+}\StringTok{ }\NormalTok{gender }\OperatorTok{+}\StringTok{ }\NormalTok{severe_disease }\OperatorTok{+}
\StringTok{    }\NormalTok{total_admission }\OperatorTok{+}\StringTok{ }\NormalTok{admdays }\OperatorTok{+}\StringTok{ }\NormalTok{nweight }\OperatorTok{+}
\StringTok{    }\KeywordTok{f}\NormalTok{(Adj_ID, }\DataTypeTok{model =} \StringTok{"bym"}\NormalTok{  ,}\DataTypeTok{graph=}\NormalTok{klf.adj , }\DataTypeTok{scale.model=}\OtherTok{TRUE}\NormalTok{,}
      \DataTypeTok{hyper=}\KeywordTok{list}\NormalTok{(}\DataTypeTok{prec.unstruct=}\KeywordTok{list}\NormalTok{(}\DataTypeTok{prior=}\StringTok{"loggamma"}\NormalTok{,}\DataTypeTok{param=}\KeywordTok{c}\NormalTok{(}\FloatTok{0.0111}\NormalTok{,}\FloatTok{0.001}\NormalTok{)),}
                 \DataTypeTok{prec.spatial=}\KeywordTok{list}\NormalTok{(}\DataTypeTok{prior=}\StringTok{"loggamma"}\NormalTok{,}\DataTypeTok{param=}\KeywordTok{c}\NormalTok{(}\FloatTok{0.0011}\NormalTok{,}\FloatTok{0.001}\NormalTok{)))) }
  

\NormalTok{  resultUHB <-}\StringTok{ }\KeywordTok{inla}\NormalTok{(formulaUHB,}\DataTypeTok{family=}\StringTok{"nbinomial"}\NormalTok{,}
                   \DataTypeTok{data=}\NormalTok{admData2, }\DataTypeTok{control.compute=}\KeywordTok{list}\NormalTok{(}\DataTypeTok{dic=}\OtherTok{TRUE}\NormalTok{,}\DataTypeTok{cpo=}\OtherTok{TRUE}\NormalTok{),}\DataTypeTok{E=}\KeywordTok{log}\NormalTok{(nagem)}
\NormalTok{                   ,}\KeywordTok{control.predictor}\NormalTok{(}\DataTypeTok{compute=}\OtherTok{TRUE}\NormalTok{))}
  \KeywordTok{summary}\NormalTok{(resultUHB)}
\NormalTok{  pdresultUHB <-}\StringTok{ }\NormalTok{resultUHB}\OperatorTok{$}\NormalTok{dic}\OperatorTok{$}\NormalTok{p.eff }\CommentTok{#25.03}
  \KeywordTok{exp}\NormalTok{(resultUHB}\OperatorTok{$}\NormalTok{summary.fixed)}
  \KeywordTok{write.csv}\NormalTok{(}\KeywordTok{data.frame}\NormalTok{(resultUHB}\OperatorTok{$}\NormalTok{summary.fixed), }\StringTok{"results2_14498.08.csv"}\NormalTok{)}
  
  \CommentTok{#write.csv(data.frame(resultUHB$summary.fixed), "results_20.05_under5_10700.63.csv")}
  
\NormalTok{  ####The   computation of the posterior mean for the random effects 𝝃 is performed in two}
  \CommentTok{# steps as we have more than one parameter:}
  \CommentTok{# we extract the marginal posterior distribution for each element of the random effect}
\NormalTok{  csi <-}\StringTok{ }\NormalTok{resultUHB}\OperatorTok{$}\NormalTok{marginals.random}\OperatorTok{$}\NormalTok{Adj_ID[}\DecValTok{1}\OperatorTok{:}\DecValTok{40}\NormalTok{]}
  
\NormalTok{  ## then apply the exponential transformation and calculate the posterior mean for each of   them using the lapply function.}
\NormalTok{  zeta <-}\StringTok{ }\KeywordTok{lapply}\NormalTok{(csi,}\ControlFlowTok{function}\NormalTok{(x) }\KeywordTok{inla.emarginal}\NormalTok{(exp,x))}
\NormalTok{  ##define the cut offs for your risk ratio}
\NormalTok{  zeta.cutoff <-}\StringTok{ }\KeywordTok{c}\NormalTok{(}\FloatTok{0.9}\NormalTok{, }\FloatTok{0.95}\NormalTok{, }\FloatTok{0.999}\NormalTok{ ,}\FloatTok{1.0}\NormalTok{,}\FloatTok{1.01}\NormalTok{,}\FloatTok{1.05}\NormalTok{, }\FloatTok{1.1}\NormalTok{)}
  
  \CommentTok{#Transform zeta in categorical variable}
\NormalTok{  cat.zeta <-}\StringTok{ }\KeywordTok{cut}\NormalTok{(}\KeywordTok{unlist}\NormalTok{(zeta),}\DataTypeTok{breaks=}\NormalTok{zeta.cutoff,}
                  \DataTypeTok{include.lowest=}\OtherTok{TRUE}\NormalTok{ )}
  
  \CommentTok{#Create a dataframe with all the information needed for the map}
\NormalTok{  maps.cat.zeta <-}\StringTok{ }\KeywordTok{data.frame}\NormalTok{(}\KeywordTok{unique}\NormalTok{(admData2}\OperatorTok{$}\NormalTok{Adj_ID), }\DataTypeTok{cat.zeta=}\NormalTok{cat.zeta)}
  
  \CommentTok{#Add the categorized zeta to the kilifi spatial polygon}
\NormalTok{  ## }
\NormalTok{  data.kilifi <-}\StringTok{ }\KeywordTok{attr}\NormalTok{(kilifi_sub, }\StringTok{"data"}\NormalTok{)}
  \KeywordTok{attr}\NormalTok{(kilifi_sub, }\StringTok{"data"}\NormalTok{) <-}\StringTok{ }\KeywordTok{merge}\NormalTok{(data.kilifi, maps.cat.zeta,}
                                    \DataTypeTok{by.x=}\StringTok{"Adj_ID"}\NormalTok{ , }\DataTypeTok{by.y=}\StringTok{"unique.admData2.Adj_ID."}\NormalTok{)}
  
\NormalTok{  ## mapping the risk ratio }
  \CommentTok{#spplot(obj=kilifi_sub, zcol= "cat.zeta", col.regions=gray(seq(0.9,0.1,length=4)), asp=1)}
  \KeywordTok{spplot}\NormalTok{(}\DataTypeTok{obj=}\NormalTok{kilifi_sub, }\DataTypeTok{zcol=} \StringTok{"cat.zeta"}\NormalTok{,}\DataTypeTok{col.regions=}\KeywordTok{diverge_hsv}\NormalTok{(}\DecValTok{8}\NormalTok{), }\DataTypeTok{scales=}\KeywordTok{list}\NormalTok{(}\DataTypeTok{draw =} \OtherTok{TRUE}\NormalTok{), }\DataTypeTok{asp=}\DecValTok{1}\NormalTok{)}
  
  
\CommentTok{#'         \textbackslash{}\textbackslash{}\textbackslash{}_Model 2\textbackslash{}\textbackslash{}\textbackslash{}         #  }
\NormalTok{###############MOdel 2#######################################################  }
\NormalTok{### spatial model structured and unstrustured  with the temporal component included}
\NormalTok{### fitting model 1 }
\NormalTok{admData2}\OperatorTok{$}\NormalTok{nagem_int <-}\StringTok{ }\KeywordTok{as.integer}\NormalTok{(admData2}\OperatorTok{$}\NormalTok{nagem)}
  
\NormalTok{formulaUH <-}\StringTok{ }\NormalTok{cumulitive_count }\OperatorTok{~}\StringTok{ }\NormalTok{EVI_VALUE }\OperatorTok{+}\StringTok{ }\NormalTok{rain_mm   }\OperatorTok{+}\StringTok{ }\NormalTok{gender }\OperatorTok{+}\StringTok{ }\NormalTok{severe_disease }\OperatorTok{+}\StringTok{ }\NormalTok{total_admission }\OperatorTok{+}\StringTok{ }\NormalTok{admdays }\OperatorTok{+}\StringTok{ }\NormalTok{nweight }\OperatorTok{+}
\StringTok{    }\KeywordTok{f}\NormalTok{(Adj_ID, }\DataTypeTok{model =} \StringTok{"bym"}\NormalTok{  ,}\DataTypeTok{graph=}\NormalTok{klf.adj , }\DataTypeTok{scale.model=}\OtherTok{TRUE}\NormalTok{,}\DataTypeTok{hyper=}\KeywordTok{list}\NormalTok{(}\DataTypeTok{prec.unstruct=}\KeywordTok{list}\NormalTok{(}\DataTypeTok{prior=}\StringTok{"loggamma"}\NormalTok{,}\DataTypeTok{param=}\KeywordTok{c}\NormalTok{(}\FloatTok{0.0111}\NormalTok{,}\FloatTok{0.001}\NormalTok{)),}
                 \DataTypeTok{prec.spatial=}\KeywordTok{list}\NormalTok{(}\DataTypeTok{prior=}\StringTok{"loggamma"}\NormalTok{,}\DataTypeTok{param=}\KeywordTok{c}\NormalTok{(}\FloatTok{0.0011}\NormalTok{,}\FloatTok{0.001}\NormalTok{)))) }\OperatorTok{+}\StringTok{ }\KeywordTok{f}\NormalTok{(count_adm, }\DataTypeTok{model =} \StringTok{"ar1"}\NormalTok{)}
  
\CommentTok{# f(count_adm, model = "ar1", replicate = Adj_ID3)}
\NormalTok{resultUH <-}\StringTok{ }\KeywordTok{inla}\NormalTok{(formulaUH,}\DataTypeTok{family=}\StringTok{"nbinomial"}\NormalTok{,}
                 \DataTypeTok{data=}\NormalTok{admData2, }\DataTypeTok{control.compute=}\KeywordTok{list}\NormalTok{(}\DataTypeTok{dic=}\OtherTok{TRUE}\NormalTok{,}\DataTypeTok{cpo=}\OtherTok{TRUE}\NormalTok{),}\DataTypeTok{E=}\KeywordTok{log}\NormalTok{(nagem_int)}
\NormalTok{                 ,}\KeywordTok{control.predictor}\NormalTok{(}\DataTypeTok{compute=}\OtherTok{TRUE}\NormalTok{))}
\KeywordTok{summary}\NormalTok{(resultUH)}
\NormalTok{pdresultUH <-}\StringTok{ }\NormalTok{resultUH}\OperatorTok{$}\NormalTok{dic}\OperatorTok{$}\NormalTok{p.eff }\CommentTok{#35.50}
\KeywordTok{exp}\NormalTok{(resultUH}\OperatorTok{$}\NormalTok{summary.fixed)}
\KeywordTok{write.csv}\NormalTok{(}\KeywordTok{data.frame}\NormalTok{(resultUH}\OperatorTok{$}\NormalTok{summary.fixed), }\StringTok{"nm_results2_13640.2.csv"}\NormalTok{)}


\NormalTok{####The   computation of the posterior mean for the random effects 𝝃 is performed in two}
\CommentTok{# steps as we have more than one parameter:}
\CommentTok{# we extract the marginal posterior distribution for each element of the random effect}
\NormalTok{csi <-}\StringTok{ }\NormalTok{resultUH}\OperatorTok{$}\NormalTok{marginals.random}\OperatorTok{$}\NormalTok{Adj_ID[}\DecValTok{1}\OperatorTok{:}\DecValTok{40}\NormalTok{]}

\NormalTok{## then apply the exponential transformation and calculate the posterior mean for each of   them using the lapply function.}
\NormalTok{zeta <-}\StringTok{ }\KeywordTok{lapply}\NormalTok{(csi,}\ControlFlowTok{function}\NormalTok{(x) }\KeywordTok{inla.emarginal}\NormalTok{(exp,x))}
\NormalTok{##define the cut offs for your risk ratio}
\NormalTok{zeta.cutoff <-}\StringTok{ }\KeywordTok{c}\NormalTok{(}\FloatTok{0.83}\NormalTok{,}\FloatTok{0.9}\NormalTok{, }\FloatTok{0.95}\NormalTok{, }\FloatTok{0.999}\NormalTok{ ,}\FloatTok{1.0}\NormalTok{,}\FloatTok{1.01}\NormalTok{,}\FloatTok{1.05}\NormalTok{, }\FloatTok{1.1}\NormalTok{ ,}\FloatTok{1.2}\NormalTok{)}

\CommentTok{#Transform zeta in categorical variable}
\NormalTok{cat.zeta <-}\StringTok{ }\KeywordTok{cut}\NormalTok{(}\KeywordTok{unlist}\NormalTok{(zeta),}\DataTypeTok{breaks=}\NormalTok{zeta.cutoff,}
                \DataTypeTok{include.lowest=}\OtherTok{TRUE}\NormalTok{ )}

\CommentTok{#Create a dataframe with all the information needed for the map}
\NormalTok{maps.cat.zeta <-}\StringTok{ }\KeywordTok{data.frame}\NormalTok{(}\KeywordTok{unique}\NormalTok{(admData2}\OperatorTok{$}\NormalTok{Adj_ID), }\DataTypeTok{cat.zeta=}\NormalTok{cat.zeta)}

\CommentTok{#Add the categorized zeta to the kilifi spatial polygon}
\NormalTok{## }
\NormalTok{data.kilifi <-}\StringTok{ }\KeywordTok{attr}\NormalTok{(kilifi_sub, }\StringTok{"data"}\NormalTok{)}
\KeywordTok{attr}\NormalTok{(kilifi_sub, }\StringTok{"data"}\NormalTok{) <-}\StringTok{ }\KeywordTok{merge}\NormalTok{(data.kilifi, maps.cat.zeta,}
                                  \DataTypeTok{by.x=}\StringTok{"Adj_ID"}\NormalTok{ , }\DataTypeTok{by.y=}\StringTok{"unique.admData2.Adj_ID."}\NormalTok{)}

\NormalTok{## mapping the risk ratio }

\KeywordTok{png}\NormalTok{(}\DataTypeTok{filename=}\KeywordTok{paste0}\NormalTok{(}\StringTok{"figure4A"}\NormalTok{,}\StringTok{"img.png"}\NormalTok{) , }\DataTypeTok{width =} \FloatTok{19.45}\NormalTok{ , }\DataTypeTok{height =}  \FloatTok{22.40}\NormalTok{ , }\DataTypeTok{units =} \StringTok{"cm"}\NormalTok{ , }\DataTypeTok{res=}\DecValTok{300}\NormalTok{)}

\KeywordTok{spplot}\NormalTok{(}\DataTypeTok{obj=}\NormalTok{kilifi_sub, }\DataTypeTok{zcol=} \StringTok{"cat.zeta"}\NormalTok{,}\DataTypeTok{col.regions=}\KeywordTok{diverge_hsv}\NormalTok{(}\DecValTok{8}\NormalTok{), }\DataTypeTok{scales=}\KeywordTok{list}\NormalTok{(}\DataTypeTok{draw =} \OtherTok{TRUE}\NormalTok{), }\DataTypeTok{asp=}\DecValTok{1}\NormalTok{)}
\KeywordTok{dev.off}\NormalTok{()}

\NormalTok{### temporal graph}

\KeywordTok{plot}\NormalTok{( resultUH, }\DataTypeTok{plot.fixed.effects =} \OtherTok{TRUE}\NormalTok{, }\DataTypeTok{constant=}\OtherTok{FALSE}\NormalTok{,}
      \DataTypeTok{plot.lincomb =} \OtherTok{TRUE}\NormalTok{, }
      \DataTypeTok{plot.random.effects =} \OtherTok{TRUE}\NormalTok{, }
      \DataTypeTok{plot.hyperparameters =} \OtherTok{TRUE}\NormalTok{,}
      \DataTypeTok{plot.predictor =} \OtherTok{TRUE}\NormalTok{, }
      \DataTypeTok{plot.q =} \OtherTok{TRUE}\NormalTok{, }
      \DataTypeTok{plot.cpo =} \OtherTok{TRUE}\NormalTok{,}
      \DataTypeTok{single =} \OtherTok{TRUE}\NormalTok{)}

  \KeywordTok{plot}\NormalTok{( resultUH, }\DataTypeTok{plot.fixed.effects =} \OtherTok{TRUE}\NormalTok{ , }\DataTypeTok{constant=}\OtherTok{FALSE}\NormalTok{,}\DataTypeTok{plot.cpo =}\NormalTok{ F,}\DataTypeTok{single =}\NormalTok{F)}
  
\KeywordTok{save.image}\NormalTok{(}\StringTok{"stModel.RDA"}\NormalTok{)  }

\CommentTok{#'         \textbackslash{}\textbackslash{}\textbackslash{}_Model 3\textbackslash{}\textbackslash{}\textbackslash{}         #}
\CommentTok{#'         ###############MOdel With variables changing over time#######################################################  }
\NormalTok{#### Fitting a SPATIAL Temporal Model}
\NormalTok{formulaUH2b <-}\StringTok{ }\NormalTok{cumulitive_count }\OperatorTok{~}\StringTok{  }\NormalTok{EVI_VALUE  }\OperatorTok{+}\StringTok{ }\NormalTok{gender }\OperatorTok{+}\StringTok{ }
\StringTok{  }\NormalTok{severe_disease }\OperatorTok{+}\StringTok{ }\NormalTok{total_admission }\OperatorTok{+}\StringTok{ }\NormalTok{rain_mm }\OperatorTok{+}\StringTok{ }\NormalTok{admdays }\OperatorTok{+}\StringTok{ }\NormalTok{nweight }\OperatorTok{+}
\StringTok{  }\KeywordTok{f}\NormalTok{(Adj_ID, }\DataTypeTok{model =} \StringTok{"bym"}\NormalTok{  ,}\DataTypeTok{graph=}\NormalTok{klf.adj , }\DataTypeTok{scale.model=}\OtherTok{TRUE}\NormalTok{,}
    \DataTypeTok{hyper=}\KeywordTok{list}\NormalTok{(}\DataTypeTok{prec.unstruct=}\KeywordTok{list}\NormalTok{(}\DataTypeTok{prior=}\StringTok{"loggamma"}\NormalTok{,}\DataTypeTok{param=}\KeywordTok{c}\NormalTok{(}\FloatTok{0.001}\NormalTok{,}\FloatTok{0.001}\NormalTok{)),}
               \DataTypeTok{prec.spatial=}\KeywordTok{list}\NormalTok{(}\DataTypeTok{prior=}\StringTok{"loggamma"}\NormalTok{,}\DataTypeTok{param=}\KeywordTok{c}\NormalTok{(}\FloatTok{0.1}\NormalTok{,}\FloatTok{0.01}\NormalTok{))))}\OperatorTok{+}\StringTok{ }
\StringTok{  }\KeywordTok{f}\NormalTok{(EVI_VALUE2 , count_adm2, }\DataTypeTok{model =} \StringTok{"iid"}\NormalTok{) }\OperatorTok{+}
\StringTok{  }\KeywordTok{f}\NormalTok{(rain_mm2 , count_adm3, }\DataTypeTok{model =} \StringTok{"iid"}\NormalTok{) }\OperatorTok{+}
\StringTok{  }\KeywordTok{f}\NormalTok{(nweight2 , count_adm5, }\DataTypeTok{model =} \StringTok{"iid"}\NormalTok{) }\OperatorTok{+}
\StringTok{  }\KeywordTok{f}\NormalTok{(admdays2 , count_adm6, }\DataTypeTok{model =} \StringTok{"iid"}\NormalTok{) }\OperatorTok{+}
\StringTok{  }\KeywordTok{f}\NormalTok{( count_adm7, }\DataTypeTok{model =} \StringTok{"ar1"}\NormalTok{)}

\NormalTok{### added due to heissan values errors }
\NormalTok{##https://groups.google.com/forum/#!topic/r-inla-discussion-group/rTdjAnILdnM}
\NormalTok{resultUH2b <-}\StringTok{ }\KeywordTok{inla}\NormalTok{(formulaUH2b,}\DataTypeTok{family=}\StringTok{"nbinomial"}\NormalTok{,}
                   \DataTypeTok{data=}\NormalTok{admData2, }\DataTypeTok{control.compute=}\KeywordTok{list}\NormalTok{(}\DataTypeTok{dic=}\OtherTok{TRUE}\NormalTok{),}\KeywordTok{control.predictor}\NormalTok{(}\DataTypeTok{compute=}\OtherTok{TRUE}\NormalTok{) ,}
                   \DataTypeTok{control.inla =} \KeywordTok{list}\NormalTok{(}\DataTypeTok{tolerance =} \FloatTok{1e-20}\NormalTok{, }\DataTypeTok{h =} \FloatTok{1e-08}\NormalTok{),}\DataTypeTok{E=}\KeywordTok{log}\NormalTok{(nagem))}
\NormalTok{pdresultH2 <-}\StringTok{ }\NormalTok{resultUH2b}\OperatorTok{$}\NormalTok{dic}\OperatorTok{$}\NormalTok{p.eff }\CommentTok{#447.2864}


\KeywordTok{summary}\NormalTok{(resultUH2b)}
\NormalTok{pdresultUH2b <-}\StringTok{ }\NormalTok{resultUH2b}\OperatorTok{$}\NormalTok{dic}\OperatorTok{$}\NormalTok{p.eff}
\KeywordTok{write.csv}\NormalTok{(}\KeywordTok{data.frame}\NormalTok{(resultUH2b}\OperatorTok{$}\NormalTok{summary.fixed), }\StringTok{"resultsST_10296.73.csv"}\NormalTok{)}

\NormalTok{csi2 <-}\StringTok{ }\NormalTok{resultUH2b}\OperatorTok{$}\NormalTok{marginals.random}\OperatorTok{$}\NormalTok{Adj_ID[}\DecValTok{1}\OperatorTok{:}\DecValTok{40}\NormalTok{]}

\NormalTok{## then apply the exponential transformation and calculate the posterior mean for each of   them using the lapply function.}
\NormalTok{zeta2 <-}\StringTok{ }\KeywordTok{lapply}\NormalTok{(csi2,}\ControlFlowTok{function}\NormalTok{(x) }\KeywordTok{inla.emarginal}\NormalTok{(exp,x))}

\NormalTok{##define the cut offs for your risk ratio}
\NormalTok{zeta.cutoff2 <-}\StringTok{ }\KeywordTok{c}\NormalTok{(}\FloatTok{0.8}\NormalTok{,}\FloatTok{0.99}\NormalTok{, }\FloatTok{1.0}\NormalTok{,}\FloatTok{1.001}\NormalTok{,}\FloatTok{1.1}\NormalTok{, }\FloatTok{1.2}\NormalTok{)}


\CommentTok{#Transform zeta in categorical variable}
\NormalTok{cat.zeta2 <-}\StringTok{ }\KeywordTok{cut}\NormalTok{(}\KeywordTok{unlist}\NormalTok{(zeta2),}\DataTypeTok{breaks=}\NormalTok{zeta.cutoff2,}
                \DataTypeTok{include.lowest=}\OtherTok{TRUE}\NormalTok{)}

\CommentTok{#Create a dataframe with all the information needed for the map}
\NormalTok{maps.cat.zeta2 <-}\StringTok{ }\KeywordTok{data.frame}\NormalTok{(}\KeywordTok{unique}\NormalTok{(admData2}\OperatorTok{$}\NormalTok{Adj_ID), }\DataTypeTok{cat.zeta2=}\NormalTok{cat.zeta2)}

\CommentTok{#Add the categorized zeta to the kilifi spatial polygon}
\NormalTok{## }
\NormalTok{data.kilifi2 <-}\StringTok{ }\KeywordTok{attr}\NormalTok{(kilifi_sub, }\StringTok{"data"}\NormalTok{)}
\KeywordTok{attr}\NormalTok{(kilifi_sub, }\StringTok{"data"}\NormalTok{) <-}\StringTok{ }\KeywordTok{merge}\NormalTok{(data.kilifi2, maps.cat.zeta2,}
                                  \DataTypeTok{by.x=}\StringTok{"Adj_ID"}\NormalTok{ , }\DataTypeTok{by.y=}\StringTok{"unique.admData2.Adj_ID."}\NormalTok{)}

\NormalTok{## mapping the risk ratio }
\KeywordTok{spplot}\NormalTok{(}\DataTypeTok{obj=}\NormalTok{kilifi_sub, }\DataTypeTok{zcol=} \StringTok{"cat.zeta2"}\NormalTok{,}\DataTypeTok{col.regions=}\KeywordTok{diverge_hsv}\NormalTok{(}\DecValTok{8}\NormalTok{), }\DataTypeTok{scales=}\KeywordTok{list}\NormalTok{(}\DataTypeTok{draw =} \OtherTok{TRUE}\NormalTok{), }\DataTypeTok{asp=}\DecValTok{1}\NormalTok{)}

\NormalTok{### temporal graph}
\KeywordTok{plot}\NormalTok{( resultUH2b, }\DataTypeTok{plot.fixed.effects =} \OtherTok{TRUE}\NormalTok{ , }\DataTypeTok{constant=}\OtherTok{FALSE}\NormalTok{,}\DataTypeTok{plot.cpo =}\NormalTok{ F,}\DataTypeTok{single =}\NormalTok{F)}
\KeywordTok{plot}\NormalTok{( resultUH2b, }\DataTypeTok{plot.fixed.effects =} \OtherTok{TRUE}\NormalTok{, }\DataTypeTok{constant=}\OtherTok{FALSE}\NormalTok{,}
      \DataTypeTok{plot.lincomb =} \OtherTok{TRUE}\NormalTok{, }
      \DataTypeTok{plot.random.effects =} \OtherTok{TRUE}\NormalTok{, }
      \DataTypeTok{plot.hyperparameters =} \OtherTok{TRUE}\NormalTok{,}
      \DataTypeTok{plot.predictor =} \OtherTok{TRUE}\NormalTok{, }
      \DataTypeTok{plot.q =} \OtherTok{TRUE}\NormalTok{, }
      \DataTypeTok{plot.cpo =} \OtherTok{TRUE}\NormalTok{,}
      \DataTypeTok{single =} \OtherTok{TRUE}\NormalTok{)}
\end{Highlighting}
\end{Shaded}

\subsection{Code 3: Temporal graphs for Spatial
Temporal}\label{code-3-temporal-graphs-for-spatial-temporal}

This code is for plotting the temporal graphs for the repeated
admission,we fit the model in a loop for each admission and also as a
temporal effect on INLA

\begin{Shaded}
\begin{Highlighting}[]
\NormalTok{### this is a loop fitting a model for each admission count and saving the graph for each admisison count }
\ControlFlowTok{for}\NormalTok{ (i }\ControlFlowTok{in} \KeywordTok{unique}\NormalTok{(admData2}\OperatorTok{$}\NormalTok{count_adm)) \{}
  \KeywordTok{rm}\NormalTok{(}\DataTypeTok{list =} \KeywordTok{ls}\NormalTok{())}
\NormalTok{  ###reading and exporting the shape file}
\NormalTok{  kilifi_sub <-}\StringTok{ }\NormalTok{maptools}\OperatorTok{::}\KeywordTok{readShapePoly}\NormalTok{ ( }\StringTok{"data/kilif_sub_loc_Shape/DSS_subloc_Arc.shp"}\NormalTok{,}
                                          \DataTypeTok{IDvar=}\StringTok{"Adj_ID"}\NormalTok{, }\DataTypeTok{proj4string=}\KeywordTok{CRS}\NormalTok{(}\StringTok{"+proj=longlat +ellps=clrk66"}\NormalTok{))}
  
\NormalTok{  temp <-}\StringTok{ }\NormalTok{spdep}\OperatorTok{::}\KeywordTok{poly2nb}\NormalTok{(kilifi_sub)}
  \KeywordTok{nb2INLA}\NormalTok{(}\StringTok{"data/kilif_sub_loc_Shape/DSS_subloc_Arc.graph"}\NormalTok{, temp)}
\NormalTok{  klf.adj <-}\StringTok{ }\KeywordTok{paste}\NormalTok{(}\KeywordTok{getwd}\NormalTok{(),}\StringTok{"/data/kilif_sub_loc_Shape/DSS_subloc_Arc.graph"}\NormalTok{,}\DataTypeTok{sep=}\StringTok{""}\NormalTok{)}
  
\NormalTok{  ### load the admissions data}
\NormalTok{  admData <-}\StringTok{ }\KeywordTok{read.csv}\NormalTok{(}\StringTok{"data/morbidity.csv"}\NormalTok{)}
\NormalTok{  admData}\OperatorTok{$}\NormalTok{rain_mm <-}\StringTok{ }\NormalTok{admData}\OperatorTok{$}\NormalTok{rain_mm}\OperatorTok{/}\DecValTok{50}
\NormalTok{  admData}\OperatorTok{$}\NormalTok{severe_disease <-}\StringTok{ }\KeywordTok{factor}\NormalTok{(admData}\OperatorTok{$}\NormalTok{severe_disease , }\DataTypeTok{levels=}\KeywordTok{c}\NormalTok{(}\DecValTok{0}\NormalTok{,}\DecValTok{1}\NormalTok{,}\DecValTok{2}\NormalTok{,}\DecValTok{3}\NormalTok{))}
\NormalTok{  admData <-}\StringTok{ }\NormalTok{admData }\OperatorTok\StringTok{ }\KeywordTok{mutate}\NormalTok{(}\DataTypeTok{gender2=} \KeywordTok{ifelse}\NormalTok{(gender}\OperatorTok{==}\DecValTok{1}\NormalTok{ ,}\DecValTok{0}\NormalTok{,}\DecValTok{1}\NormalTok{ ))}
  
\NormalTok{  admData}\OperatorTok{$}\NormalTok{gender2 <-}\StringTok{ }\KeywordTok{factor}\NormalTok{(admData}\OperatorTok{$}\NormalTok{gender2 , }\DataTypeTok{levels=}\KeywordTok{c}\NormalTok{(}\DecValTok{0}\NormalTok{,}\DecValTok{1}\NormalTok{) )}
\NormalTok{  admData}\OperatorTok{$}\NormalTok{gender <-}\StringTok{ }\NormalTok{admData}\OperatorTok{$}\NormalTok{gender2}
\NormalTok{  admData2 <-}\StringTok{ }\NormalTok{admData }\OperatorTok\StringTok{  }\NormalTok{dplyr}\OperatorTok{::}\KeywordTok{select}\NormalTok{(Adj_ID , sublocation,mnth, nagem, gender , }
\NormalTok{                                         severe_disease ,}
\NormalTok{                                         cumulitive_count,cumulitive_time , EVI_VALUE ,count_adm ,rain_mm,}
\NormalTok{                                         total_admission ,admdays ,nweight ,yr)}
  
\NormalTok{  ###}
\NormalTok{  admData2}\OperatorTok{$}\NormalTok{Adj_ID2 <-}\StringTok{ }\NormalTok{admData2}\OperatorTok{$}\NormalTok{Adj_ID}
\NormalTok{  admData2}\OperatorTok{$}\NormalTok{Adj_ID3 <-}\StringTok{ }\NormalTok{admData2}\OperatorTok{$}\NormalTok{Adj_ID}
\NormalTok{  admData2}\OperatorTok{$}\NormalTok{count_adm2 <-}\StringTok{ }\NormalTok{admData2}\OperatorTok{$}\NormalTok{count_adm}
\NormalTok{  admData2}\OperatorTok{$}\NormalTok{count_adm3 <-}\StringTok{ }\NormalTok{admData2}\OperatorTok{$}\NormalTok{count_adm}
\NormalTok{  admData2}\OperatorTok{$}\NormalTok{count_adm4 <-}\StringTok{ }\NormalTok{admData2}\OperatorTok{$}\NormalTok{count_adm}
\NormalTok{  admData2}\OperatorTok{$}\NormalTok{count_adm5 <-}\StringTok{ }\NormalTok{admData2}\OperatorTok{$}\NormalTok{count_adm}
\NormalTok{  admData2}\OperatorTok{$}\NormalTok{count_adm6 <-}\StringTok{ }\NormalTok{admData2}\OperatorTok{$}\NormalTok{count_adm}
\NormalTok{  admData2}\OperatorTok{$}\NormalTok{count_adm7 <-}\StringTok{ }\NormalTok{admData2}\OperatorTok{$}\NormalTok{count_adm}
\NormalTok{  admData2}\OperatorTok{$}\NormalTok{EVI_VALUE2 <-}\StringTok{ }\NormalTok{admData2}\OperatorTok{$}\NormalTok{EVI_VALUE}
\NormalTok{  admData2}\OperatorTok{$}\NormalTok{rain_mm2 <-}\StringTok{ }\NormalTok{admData2}\OperatorTok{$}\NormalTok{rain_mm}
\NormalTok{  admData2}\OperatorTok{$}\NormalTok{nagem2 <-}\StringTok{ }\NormalTok{admData2}\OperatorTok{$}\NormalTok{nagem}
\NormalTok{  admData2}\OperatorTok{$}\NormalTok{severe_disease2 <-}\StringTok{ }\KeywordTok{as.factor}\NormalTok{(admData2}\OperatorTok{$}\NormalTok{severe_disease)}
\NormalTok{  admData2}\OperatorTok{$}\NormalTok{mnth2 <-}\StringTok{ }\NormalTok{admData2}\OperatorTok{$}\NormalTok{mnth}
\NormalTok{  admData2}\OperatorTok{$}\NormalTok{nweight2 <-}\StringTok{   }\NormalTok{admData2}\OperatorTok{$}\NormalTok{nweight}
\NormalTok{  admData2}\OperatorTok{$}\NormalTok{admdays2 <-}\StringTok{   }\NormalTok{admData2}\OperatorTok{$}\NormalTok{admdays}
  
  
\NormalTok{  admData2x <-}\StringTok{ }\NormalTok{admData2 }\OperatorTok\StringTok{ }\KeywordTok{filter}\NormalTok{(count_adm}\OperatorTok{==}\NormalTok{i)}
\NormalTok{  formulaUH <-}\StringTok{ }\NormalTok{cumulitive_count }\OperatorTok{~}\StringTok{ }\NormalTok{EVI_VALUE }\OperatorTok{+}\StringTok{ }\NormalTok{rain_mm   }\OperatorTok{+}\StringTok{ }
\StringTok{  }\NormalTok{gender }\OperatorTok{+}\StringTok{ }\NormalTok{severe_disease }\OperatorTok{+}\StringTok{ }\NormalTok{total_admission }\OperatorTok{+}\StringTok{ }\NormalTok{admdays }\OperatorTok{+}\StringTok{ }\NormalTok{nweight }\OperatorTok{+}
\StringTok{  }\KeywordTok{f}\NormalTok{(Adj_ID, }\DataTypeTok{model =} \StringTok{"bym"}\NormalTok{  ,}\DataTypeTok{graph=}\NormalTok{klf.adj , }\DataTypeTok{scale.model=}\OtherTok{TRUE}\NormalTok{,}
    \DataTypeTok{hyper=}\KeywordTok{list}\NormalTok{(}\DataTypeTok{prec.unstruct=}\KeywordTok{list}\NormalTok{(}\DataTypeTok{prior=}\StringTok{"loggamma"}\NormalTok{,}\DataTypeTok{param=}\KeywordTok{c}\NormalTok{(}\DecValTok{1}\NormalTok{,}\FloatTok{0.001}\NormalTok{)),}
               \DataTypeTok{prec.spatial=}\KeywordTok{list}\NormalTok{(}\DataTypeTok{prior=}\StringTok{"loggamma"}\NormalTok{,}\DataTypeTok{param=}\KeywordTok{c}\NormalTok{(}\DecValTok{1}\NormalTok{,}\FloatTok{0.001}\NormalTok{)))) }

\CommentTok{# f(count_adm, model = "ar1", replicate = Adj_ID3)}
\NormalTok{  resultUH <-}\StringTok{ }\KeywordTok{inla}\NormalTok{(formulaUH,}\DataTypeTok{family=}\StringTok{"nbinomial"}\NormalTok{,}
                 \DataTypeTok{data=}\NormalTok{admData2x, }\DataTypeTok{control.compute=}\KeywordTok{list}\NormalTok{(}\DataTypeTok{dic=}\OtherTok{TRUE}\NormalTok{,}\DataTypeTok{cpo=}\OtherTok{TRUE}\NormalTok{),}\DataTypeTok{E=}\KeywordTok{log}\NormalTok{(nagem)}
\NormalTok{                 ,}\KeywordTok{control.predictor}\NormalTok{(}\DataTypeTok{compute=}\OtherTok{TRUE}\NormalTok{))}

\CommentTok{#summary(resultUH)}

\CommentTok{#exp(resultUH$summary.fixed)}
  \KeywordTok{write.csv}\NormalTok{(}\KeywordTok{data.frame}\NormalTok{(}\KeywordTok{exp}\NormalTok{(resultUH}\OperatorTok{$}\NormalTok{summary.fixed)), }\KeywordTok{paste0}\NormalTok{(i,}\StringTok{"_results2_10504.53.csv"}\NormalTok{))}
\NormalTok{####The   computation of the posterior mean for the random effects 𝝃 is performed in two}
\CommentTok{# steps as we have more than one parameter:}
\CommentTok{# we extract the marginal posterior distribution for each element of the random effect}
\NormalTok{csi <-}\StringTok{ }\NormalTok{resultUH}\OperatorTok{$}\NormalTok{marginals.random}\OperatorTok{$}\NormalTok{Adj_ID[}\DecValTok{1}\OperatorTok{:}\DecValTok{40}\NormalTok{]}

\NormalTok{## then apply the exponential transformation and calculate the posterior mean for each of   them using the lapply function.}
\NormalTok{zeta <-}\StringTok{ }\KeywordTok{lapply}\NormalTok{(csi,}\ControlFlowTok{function}\NormalTok{(x) }\KeywordTok{inla.emarginal}\NormalTok{(exp,x))}
\NormalTok{##define the cut offs for your risk ratio}
\NormalTok{zeta.cutoff <-}\StringTok{ }\KeywordTok{c}\NormalTok{(}\FloatTok{0.9}\NormalTok{, }\FloatTok{0.95}\NormalTok{, }\FloatTok{0.999}\NormalTok{ ,}\FloatTok{1.0}\NormalTok{,}\FloatTok{1.01}\NormalTok{,}\FloatTok{1.05}\NormalTok{, }\FloatTok{1.1}\NormalTok{)}

\CommentTok{#Transform zeta in categorical variable}
\NormalTok{cat.zeta <-}\StringTok{ }\KeywordTok{cut}\NormalTok{(}\KeywordTok{unlist}\NormalTok{(zeta),}\DataTypeTok{breaks=}\NormalTok{zeta.cutoff,}
                \DataTypeTok{include.lowest=}\OtherTok{TRUE}\NormalTok{ )}

\CommentTok{#Create a dataframe with all the information needed for the map}
\NormalTok{maps.cat.zeta <-}\StringTok{ }\KeywordTok{data.frame}\NormalTok{(}\KeywordTok{unique}\NormalTok{(admData2}\OperatorTok{$}\NormalTok{Adj_ID), }\DataTypeTok{cat.zeta=}\NormalTok{cat.zeta)}

\CommentTok{#Add the categorized zeta to the kilifi spatial polygon}
\NormalTok{## }
\NormalTok{data.kilifi <-}\StringTok{ }\KeywordTok{attr}\NormalTok{(kilifi_sub, }\StringTok{"data"}\NormalTok{)}
\KeywordTok{attr}\NormalTok{(kilifi_sub, }\StringTok{"data"}\NormalTok{) <-}\StringTok{ }\KeywordTok{merge}\NormalTok{(data.kilifi, maps.cat.zeta,}
                                  \DataTypeTok{by.x=}\StringTok{"Adj_ID"}\NormalTok{ , }\DataTypeTok{by.y=}\StringTok{"unique.admData2.Adj_ID."}\NormalTok{)}

\NormalTok{## mapping the risk ratio }
\KeywordTok{png}\NormalTok{(}\DataTypeTok{filename=}\KeywordTok{paste0}\NormalTok{(}\StringTok{"temp_"}\NormalTok{, i,}\StringTok{"_count.png"}\NormalTok{) , }\DataTypeTok{width =} \FloatTok{15.47}\NormalTok{ , }\DataTypeTok{height =} \FloatTok{17.57}\NormalTok{ , }\DataTypeTok{units =} \StringTok{"cm"}\NormalTok{ , }\DataTypeTok{res=}\DecValTok{72}\NormalTok{)}
\KeywordTok{spplot}\NormalTok{(}\DataTypeTok{obj=}\NormalTok{kilifi_sub, }\DataTypeTok{zcol=} \StringTok{"cat.zeta"}\NormalTok{,}\DataTypeTok{col.regions=}\KeywordTok{diverge_hsv}\NormalTok{(}\DecValTok{8}\NormalTok{), }\DataTypeTok{scales=}\KeywordTok{list}\NormalTok{(}\DataTypeTok{draw =} \OtherTok{TRUE}\NormalTok{), }\DataTypeTok{asp=}\DecValTok{1}\NormalTok{)}
\KeywordTok{dev.off}\NormalTok{()}
\NormalTok{\}}


\NormalTok{## here we fit an AR1 model with INLA and extract the plots for each timepoint}
\NormalTok{## allow for an interaction between space and time,}
\CommentTok{#which would explain differences in the time trend of malnutrition related admissions for different areas,}
\NormalTok{### Type III interaction - used in this paper to report}
\NormalTok{##Type III combines the unstructured temporal effect ????t and the spatially structured main effect ui}

\NormalTok{ID.area.int <-}\StringTok{ }\NormalTok{admData2}\OperatorTok{$}\NormalTok{Adj_ID}
\NormalTok{ID.year.int <-}\StringTok{ }\NormalTok{admData2}\OperatorTok{$}\NormalTok{count_adm}
\NormalTok{temporalModel3 <-}\StringTok{ }\NormalTok{cumulitive_count }\OperatorTok{~}\StringTok{  }\NormalTok{EVI_VALUE  }\OperatorTok{+}\StringTok{ }\NormalTok{gender }\OperatorTok{+}\StringTok{ }
\StringTok{  }\NormalTok{severe_disease }\OperatorTok{+}\StringTok{  }\NormalTok{rain_mm }\OperatorTok{+}\StringTok{ }\NormalTok{admdays }\OperatorTok{+}\StringTok{ }\NormalTok{nweight }\OperatorTok{+}
\StringTok{  }\KeywordTok{f}\NormalTok{(Adj_ID, }\DataTypeTok{model =} \StringTok{"bym"}\NormalTok{  ,}\DataTypeTok{graph=}\NormalTok{klf.adj , }\DataTypeTok{scale.model=}\OtherTok{TRUE}\NormalTok{,}
    \DataTypeTok{hyper=}\KeywordTok{list}\NormalTok{(}\DataTypeTok{prec.unstruct=}\KeywordTok{list}\NormalTok{(}\DataTypeTok{prior=}\StringTok{"loggamma"}\NormalTok{,}\DataTypeTok{param=}\KeywordTok{c}\NormalTok{(}\FloatTok{0.001}\NormalTok{,}\FloatTok{0.001}\NormalTok{)),}
               \DataTypeTok{prec.spatial=}\KeywordTok{list}\NormalTok{(}\DataTypeTok{prior=}\StringTok{"loggamma"}\NormalTok{,}\DataTypeTok{param=}\KeywordTok{c}\NormalTok{(}\FloatTok{0.1}\NormalTok{,}\FloatTok{0.01}\NormalTok{)))) }\OperatorTok{+}
\StringTok{  }\KeywordTok{f}\NormalTok{( count_adm, }\DataTypeTok{model =} \StringTok{"ar1"}\NormalTok{) }\OperatorTok{+}\StringTok{ }\KeywordTok{f}\NormalTok{(ID.year.int,}\DataTypeTok{model=}\StringTok{"iid"}\NormalTok{, }\DataTypeTok{group=}\NormalTok{ID.area.int,}
                                   \DataTypeTok{control.group=}\KeywordTok{list}\NormalTok{(}\DataTypeTok{model=}\StringTok{"besag"}\NormalTok{,}
                                                      \DataTypeTok{graph=}\NormalTok{klf.adj))}

\NormalTok{result_tM3 <-}\StringTok{ }\KeywordTok{inla}\NormalTok{(temporalModel3,}\DataTypeTok{family=}\StringTok{"nbinomial"}\NormalTok{,}
                   \DataTypeTok{data=}\NormalTok{admData2, }\DataTypeTok{control.compute=}\KeywordTok{list}\NormalTok{(}\DataTypeTok{dic=}\OtherTok{TRUE}\NormalTok{),}\KeywordTok{control.predictor}\NormalTok{(}\DataTypeTok{compute=}\OtherTok{TRUE}\NormalTok{) ,}
                   \DataTypeTok{control.inla =} \KeywordTok{list}\NormalTok{(}\DataTypeTok{tolerance =} \FloatTok{1e-20}\NormalTok{, }\DataTypeTok{h =} \FloatTok{1e-08}\NormalTok{),}\DataTypeTok{E=}\KeywordTok{log}\NormalTok{(nagem_int))}


\NormalTok{delta.intIII <-}\StringTok{ }\KeywordTok{data.frame}\NormalTok{(}\DataTypeTok{delta=}\KeywordTok{exp}\NormalTok{(result_tM3}\OperatorTok{$}\NormalTok{summary.random}\OperatorTok{$}\NormalTok{ID.year.int[,}\DecValTok{2}\NormalTok{]),}\DataTypeTok{tempC=}\KeywordTok{rep}\NormalTok{(}\DecValTok{1}\OperatorTok{:}\DecValTok{11}\NormalTok{, }\DataTypeTok{each =} \DecValTok{40}\NormalTok{),}
                           \DataTypeTok{ID.area=}\NormalTok{result_tM3}\OperatorTok{$}\NormalTok{summary.random}\OperatorTok{$}\NormalTok{ID.year.int[,}\DecValTok{1}\NormalTok{])}
\NormalTok{delta.intIII.matrix <-}\StringTok{ }\KeywordTok{matrix}\NormalTok{(delta.intIII[,}\DecValTok{1}\NormalTok{], }\DecValTok{40}\NormalTok{,}\DecValTok{11}\NormalTok{,}\DataTypeTok{byrow=}\OtherTok{FALSE}\NormalTok{)}
\KeywordTok{rownames}\NormalTok{(delta.intIII.matrix)<-}\StringTok{ }\NormalTok{delta.intIII[}\DecValTok{1}\OperatorTok{:}\DecValTok{40}\NormalTok{,}\DecValTok{3}\NormalTok{]}

\KeywordTok{save.image}\NormalTok{(}\StringTok{"st_model3.RDA"}\NormalTok{)}


\NormalTok{cutoff.interaction <-}\StringTok{ }\KeywordTok{c}\NormalTok{(}\FloatTok{0.50}\NormalTok{,}\FloatTok{0.880}\NormalTok{, }\FloatTok{1.0}\NormalTok{, }\FloatTok{1.3}\NormalTok{,}\FloatTok{1.9}\NormalTok{,}\FloatTok{2.5}\NormalTok{,}\FloatTok{3.4}\NormalTok{,}\FloatTok{7.4}\NormalTok{)}
\NormalTok{data.klf <-}\StringTok{ }\KeywordTok{attr}\NormalTok{(kilifi_sub, }\StringTok{"data"}\NormalTok{)}
\NormalTok{delta.intIII.factor <-}\StringTok{ }\KeywordTok{data.frame}\NormalTok{(}\DataTypeTok{NAME=}\NormalTok{data.klf}\OperatorTok{$}\NormalTok{Adj_ID)}
\ControlFlowTok{for}\NormalTok{(i }\ControlFlowTok{in} \DecValTok{1}\OperatorTok{:}\DecValTok{11}\NormalTok{)\{}
\NormalTok{  delta.factor.temp <-}\StringTok{ }\KeywordTok{cut}\NormalTok{(delta.intIII.matrix[,i],}\DataTypeTok{breaks=}\NormalTok{cutoff.interaction,}\DataTypeTok{include.lowest=}\OtherTok{TRUE}\NormalTok{ ) }
\NormalTok{  delta.intIII.factor <-}\StringTok{ }\KeywordTok{cbind}\NormalTok{(delta.intIII.factor,delta.factor.temp)}
\NormalTok{\}}
\KeywordTok{colnames}\NormalTok{(delta.intIII.factor)<-}\StringTok{ }\KeywordTok{c}\NormalTok{(}\StringTok{"NAME"}\NormalTok{,}\KeywordTok{seq}\NormalTok{(}\DecValTok{1}\NormalTok{,}\DecValTok{11}\NormalTok{))}

\CommentTok{# *** Code for Figure 7.6}
\KeywordTok{attr}\NormalTok{(kilifi_sub, }\StringTok{"data"}\NormalTok{) <-}\StringTok{ }\KeywordTok{data.frame}\NormalTok{(data.klf,  }\DataTypeTok{intIII=}\NormalTok{delta.intIII.factor)}
\KeywordTok{trellis.par.set}\NormalTok{(}\DataTypeTok{axis.line=}\KeywordTok{list}\NormalTok{(}\DataTypeTok{col=}\OtherTok{NA}\NormalTok{))}

\KeywordTok{png}\NormalTok{(}\DataTypeTok{filename=}\KeywordTok{paste0}\NormalTok{(}\StringTok{"temp_"}\NormalTok{,}\StringTok{"img.png"}\NormalTok{) , }\DataTypeTok{width =} \FloatTok{25.47}\NormalTok{ , }\DataTypeTok{height =} \FloatTok{27.57}\NormalTok{ , }\DataTypeTok{units =} \StringTok{"cm"}\NormalTok{ , }\DataTypeTok{res=}\DecValTok{300}\NormalTok{)}

\KeywordTok{spplot}\NormalTok{(}\DataTypeTok{obj=}\NormalTok{kilifi_sub, }\DataTypeTok{zcol=}\KeywordTok{c}\NormalTok{(}\StringTok{"intIII.1"}\NormalTok{,}\StringTok{"intIII.2"}\NormalTok{,}\StringTok{"intIII.3"}\NormalTok{,}
                              \StringTok{"intIII.4"}\NormalTok{, }\StringTok{"intIII.5"}\NormalTok{,}\StringTok{"intIII.6"}\NormalTok{,}
                              \StringTok{"intIII.7"}\NormalTok{, }\StringTok{"intIII.8"}\NormalTok{,}\StringTok{"intIII.9"}\NormalTok{,}
                              \StringTok{"intIII.10"}\NormalTok{,}\StringTok{"intIII.11"}\NormalTok{), }
       \DataTypeTok{col.regions=}\KeywordTok{diverge_hsv}\NormalTok{(}\DecValTok{8}\NormalTok{),}
       \DataTypeTok{names.attr=}\KeywordTok{seq}\NormalTok{(}\DecValTok{1}\NormalTok{,}\DecValTok{11}\NormalTok{),}\DataTypeTok{main=}\StringTok{""}\NormalTok{)    }
\KeywordTok{dev.off}\NormalTok{()}

\NormalTok{##  code for fitting with interaction 2}
\NormalTok{### Type II interaction }
\NormalTok{##Type II combines the structured temporal main effect and unstructured interactions}
\CommentTok{#f(ID.area.int,model="iid", group=ID.year.int,control.group=list(model="ar1")) }
\NormalTok{## interaction 2}
\NormalTok{ID.area.int <-}\StringTok{ }\NormalTok{admData2}\OperatorTok{$}\NormalTok{Adj_ID}
\NormalTok{ID.year.int <-}\StringTok{ }\NormalTok{admData2}\OperatorTok{$}\NormalTok{count_adm}
\NormalTok{temporalModel2 <-}\StringTok{ }\NormalTok{cumulitive_count }\OperatorTok{~}\StringTok{  }\NormalTok{EVI_VALUE  }\OperatorTok{+}\StringTok{ }\NormalTok{gender }\OperatorTok{+}\StringTok{ }
\StringTok{  }\NormalTok{severe_disease }\OperatorTok{+}\StringTok{ }\NormalTok{total_admission }\OperatorTok{+}\StringTok{ }\NormalTok{rain_mm }\OperatorTok{+}\StringTok{ }\NormalTok{admdays }\OperatorTok{+}\StringTok{ }\NormalTok{nweight }\OperatorTok{+}
\StringTok{  }\KeywordTok{f}\NormalTok{(Adj_ID, }\DataTypeTok{model =} \StringTok{"bym"}\NormalTok{  ,}\DataTypeTok{graph=}\NormalTok{klf.adj , }\DataTypeTok{scale.model=}\OtherTok{TRUE}\NormalTok{,}
    \DataTypeTok{hyper=}\KeywordTok{list}\NormalTok{(}\DataTypeTok{prec.unstruct=}\KeywordTok{list}\NormalTok{(}\DataTypeTok{prior=}\StringTok{"loggamma"}\NormalTok{,}\DataTypeTok{param=}\KeywordTok{c}\NormalTok{(}\FloatTok{0.001}\NormalTok{,}\FloatTok{0.001}\NormalTok{)),}
               \DataTypeTok{prec.spatial=}\KeywordTok{list}\NormalTok{(}\DataTypeTok{prior=}\StringTok{"loggamma"}\NormalTok{,}\DataTypeTok{param=}\KeywordTok{c}\NormalTok{(}\FloatTok{0.1}\NormalTok{,}\FloatTok{0.01}\NormalTok{)))) }\OperatorTok{+}
\StringTok{  }\KeywordTok{f}\NormalTok{( count_adm, }\DataTypeTok{model =} \StringTok{"ar1"}\NormalTok{) }\OperatorTok{+}\StringTok{ }\KeywordTok{f}\NormalTok{(ID.area.int,}\DataTypeTok{model=}\StringTok{"iid"}\NormalTok{, }\DataTypeTok{group=}\NormalTok{ID.year.int,}\DataTypeTok{control.group=}\KeywordTok{list}\NormalTok{(}\DataTypeTok{model=}\StringTok{"ar1"}\NormalTok{)) }


\NormalTok{result_tM2 <-}\StringTok{ }\KeywordTok{inla}\NormalTok{(temporalModel2,}\DataTypeTok{family=}\StringTok{"nbinomial"}\NormalTok{,}
                   \DataTypeTok{data=}\NormalTok{admData2, }\DataTypeTok{control.compute=}\KeywordTok{list}\NormalTok{(}\DataTypeTok{dic=}\OtherTok{TRUE}\NormalTok{),}\KeywordTok{control.predictor}\NormalTok{(}\DataTypeTok{compute=}\OtherTok{TRUE}\NormalTok{) ,}
                   \DataTypeTok{control.inla =} \KeywordTok{list}\NormalTok{(}\DataTypeTok{tolerance =} \FloatTok{1e-20}\NormalTok{, }\DataTypeTok{h =} \FloatTok{1e-08}\NormalTok{),}\DataTypeTok{E=}\KeywordTok{log}\NormalTok{(nagem))}

\NormalTok{### }
\NormalTok{delta.intII <-}\StringTok{ }\KeywordTok{data.frame}\NormalTok{(}\DataTypeTok{delta=}\KeywordTok{exp}\NormalTok{(result_tM2}\OperatorTok{$}\NormalTok{summary.random}\OperatorTok{$}\NormalTok{ID.area.int[,}\DecValTok{2}\NormalTok{]),}
                          \DataTypeTok{tempC=}\KeywordTok{rep}\NormalTok{(}\DecValTok{1}\OperatorTok{:}\DecValTok{11}\NormalTok{, }\DataTypeTok{each =} \DecValTok{40}\NormalTok{) ,}\DataTypeTok{ID.area=}\NormalTok{result_tM2}\OperatorTok{$}\NormalTok{summary.random}\OperatorTok{$}\NormalTok{ID.area.int[,}\DecValTok{1}\NormalTok{])}
\NormalTok{delta.intII.matrix <-}\StringTok{ }\KeywordTok{matrix}\NormalTok{(delta.intII[,}\DecValTok{1}\NormalTok{], }\DecValTok{40}\NormalTok{,}\DecValTok{11}\NormalTok{,}\DataTypeTok{byrow=}\OtherTok{FALSE}\NormalTok{)}
\KeywordTok{rownames}\NormalTok{(delta.intII.matrix)<-}\StringTok{ }\NormalTok{delta.intII[}\DecValTok{1}\OperatorTok{:}\DecValTok{40}\NormalTok{,}\DecValTok{3}\NormalTok{]}


\CommentTok{# Check the absence of spatial trend for (intII)}
\CommentTok{# cutoff.interaction <- c(-1,-0.01,0.01,1)}
\NormalTok{cutoff.interaction <-}\StringTok{ }\KeywordTok{c}\NormalTok{(}\FloatTok{0.20}\NormalTok{, }\FloatTok{0.50}\NormalTok{,}\FloatTok{0.70}\NormalTok{, }\FloatTok{0.999}\NormalTok{ ,}\FloatTok{1.0}\NormalTok{,}\FloatTok{1.01}\NormalTok{, }\FloatTok{1.1}\NormalTok{,}\FloatTok{1.4}\NormalTok{,}\FloatTok{1.7}\NormalTok{)}
\NormalTok{data.klf <-}\StringTok{ }\KeywordTok{attr}\NormalTok{(kilifi_sub, }\StringTok{"data"}\NormalTok{)}
\NormalTok{delta.intII.factor <-}\StringTok{ }\KeywordTok{data.frame}\NormalTok{(}\DataTypeTok{NAME=}\NormalTok{data.klf}\OperatorTok{$}\NormalTok{Adj_ID)}
\ControlFlowTok{for}\NormalTok{(i }\ControlFlowTok{in} \DecValTok{1}\OperatorTok{:}\DecValTok{11}\NormalTok{)\{}
\NormalTok{  delta.factor.temp <-}\StringTok{ }\KeywordTok{cut}\NormalTok{(delta.intII.matrix[,i],}\DataTypeTok{breaks=}\NormalTok{cutoff.interaction,}\DataTypeTok{include.lowest=}\OtherTok{TRUE}\NormalTok{) }
\NormalTok{  delta.intII.factor <-}\StringTok{ }\KeywordTok{cbind}\NormalTok{(delta.intII.factor,delta.factor.temp)}
\NormalTok{\}}
\KeywordTok{colnames}\NormalTok{(delta.intII.factor)<-}\StringTok{ }\KeywordTok{c}\NormalTok{(}\StringTok{"NAME"}\NormalTok{,}\KeywordTok{seq}\NormalTok{(}\DecValTok{1}\NormalTok{,}\DecValTok{11}\NormalTok{))}

\CommentTok{# *** Code for Figure 7.5}
\KeywordTok{attr}\NormalTok{(kilifi_sub, }\StringTok{"data"}\NormalTok{) <-}\StringTok{ }\KeywordTok{data.frame}\NormalTok{(data.klf,  }\DataTypeTok{intII=}\NormalTok{delta.intII.factor)}
\KeywordTok{trellis.par.set}\NormalTok{(}\DataTypeTok{axis.line=}\KeywordTok{list}\NormalTok{(}\DataTypeTok{col=}\OtherTok{NA}\NormalTok{))}

\KeywordTok{spplot}\NormalTok{(}\DataTypeTok{obj=}\NormalTok{kilifi_sub, }\DataTypeTok{zcol=}\KeywordTok{c}\NormalTok{(}\StringTok{"intII.1"}\NormalTok{,}\StringTok{"intII.2"}\NormalTok{,}\StringTok{"intII.3"}\NormalTok{,}
                              \StringTok{"intII.4"}\NormalTok{, }\StringTok{"intII.5"}\NormalTok{,}\StringTok{"intII.6"}\NormalTok{,}
                              \StringTok{"intII.7"}\NormalTok{, }\StringTok{"intII.8"}\NormalTok{,}\StringTok{"intII.9"}\NormalTok{,}
                              \StringTok{"intII.10"}\NormalTok{,}\StringTok{"intII.11"}\NormalTok{), }
       \DataTypeTok{col.regions=}\KeywordTok{diverge_hsv}\NormalTok{(}\DecValTok{8}\NormalTok{),}
       \DataTypeTok{names.attr=}\KeywordTok{seq}\NormalTok{(}\DecValTok{1}\NormalTok{,}\DecValTok{11}\NormalTok{),}\DataTypeTok{main=}\StringTok{""}\NormalTok{)     }
\end{Highlighting}
\end{Shaded}


\end{document}
